\documentclass{article}
\usepackage{ctex}
\usepackage{geometry}
\usepackage{amsfonts,amssymb, amsmath}
\usepackage{listings} 
\usepackage{fontspec}
\usepackage{graphicx}
\geometry{a4paper,left=2.5cm,right=2.5cm,top=2cm,bottom=2cm}
\newcommand*{\dif}{\mathop{}\!\mathrm{d}}
\renewcommand{\baselinestretch}{1.5}
\title{科学计算作业9}	
\author{张文涛 517030910425}
\date{\today}
\lstset{columns=flexible,numbers=left,numberstyle=\tiny,basicstyle=\small,keywordstyle=\color{blue!70},commentstyle=\color{red!50!green!50!blue!50}, rulesepcolor= \color{ red!20!green!20!blue!20} }
\begin{document}
	\maketitle
	\begin{itemize}
		\item[1.]定义函数$f:\mathbb{R} \rightarrow \mathbb{R}$如下:
		$$f(x) = \ln(1+e^x)$$
		证明:$f$在任何有限闭区间$[a,b]$上是压缩映像,但没有不动点。\\
		\\
		证:对于任何有限闭区间$[a,b]$,有:
		$$\ln(1+e^b) - \ln(1+e^a) -(b -a) = \ln(1+\frac{1}{e^b}) - \ln(1+\frac{1}{e^a}) = \ln\left(\frac{e^{a+b}+e^a}{e^{a+b} + e^b}\right)<0$$
		所以是压缩映像\\
		假设原函数有不动点$x_{*}$,代入并两边取指数得到:
		$$e^{x_*} = 1 + e^{x_*}$$
		显然$x_*$不存在,因此没有不动点。\\
		\\
		\item[2.]设$f(x)\in C^{1}(\mathbb{R})$,满足$f(x_*) = 0.$设对于一切$x\in \mathbb{R},$满足$0<m\le f'(x)\le M$,证明:若参数$\lambda\in(0,2/M)$,则迭代法
		$$x_{k+1} = x_{k} - \lambda f(x_k)$$
		取任意初值$x_0\in \mathbb{R}$产生的序列收敛到$x_{*}.$\\
		\\
		证:
		记$\phi(x) =  x - f(x)$,由题意有$\phi(x_*) = x_*$,即$x_*$为$\phi(x)$的不动点,对于$\mathbb{R}$上任意包含$x_*$的区间$[a, b]$,有:
		$$\phi(a) - \phi(b) = \phi'(\xi)(b - a) \le \max\{1 - \lambda m, 1 - \lambda M\}(b - a)$$
		$$|\phi(a) - \phi(b) |= |\phi'(\xi)||b - a| \le \max\{|1 - \lambda m|, |1 - \lambda M|\}|b - a|$$
		由于$\lambda \in(0, 2/M)$,则可以验证$|1 - \lambda m|<1, |1 - \lambda M|<1$,因此对于$x_{k+1}\,, x_{k} \in [a, b]$,都有:
		$$|x_{k+2} -  x_{k+1}| = |\phi(x_{k+1}) - \phi(x_k)| \le L|x_{k+1} - x_{k}|\qquad L = \max\{|1 - \lambda m|, |1 - \lambda M|\} \in(0,1)$$
		所以对于任意的起始值$x_0$,可以进行一次迭代得到起始区间$[x_0, x_1]$,由于我们证明了对于任意$\mathbb{R}$上区间$\phi(x)$都是压缩映射,且由不动点的唯一性可以得知$x_k$收敛于$x_*$。\\
		\\
		\item[3.]设$a>0$,$n$为正整数,应用Newton法于方程$f(x) = x^{n} - a = 0$及$g(x) = 1 - a/x^n = 0$求正根.\\
		(1)分别导出求解$\sqrt[n]{a}$的迭代表达式$x_{k+1} = \phi_{1}(x_k)$及$x_{k+1} = \phi_{2}(x_k).$\\
		(2)分别计算$\lim\limits_{k \rightarrow \infty}(\sqrt[n]{a} - x_{k+1})/(\sqrt[n]{a} - x_{k})^2.$\\
		(3)) 确定待定系数, 使得基于映射$\phi = c_1\phi_1 + c_2\phi_2 $导出的不动点方法产生的迭代序
		列 ${x_k}$ 三阶收敛到$\sqrt[n]{a}$.\\
		\\
		(1)利用牛顿迭代法公式,有:
		$$x_{k+1} = x_{k} - \frac{f(x_k)}{f'(x_k)} = x_k - \frac{x_{k}^n - a}{nx_k^{n - 1}} = \phi_{1}(x_k)$$
		$$x_{k+1} = x_{k} - \frac{f(x_k)}{f'(x_k)} = x_k - \frac{x_{k}^{n+1} - ax_k}{na} = \phi_{2}(x_k)$$\\
		\\
		(2)由题意得:
		$$\frac{\sqrt[n]{a} - x_{k+1}}{(\sqrt[n]{a} - x_{k})^2} = \frac{\phi_1(\sqrt[n]{a}) -\phi_{1}( x_{k})}{(\sqrt[n]{a} - x_{k})^2} =  \frac{\phi_{1}'(\sqrt[n]{a})(x_k  -\sqrt[n]{a}) + \frac{\phi_{1}''(\xi_1)(x_k  -\sqrt[n]{a})^2}{2!}}{(\sqrt[n]{a} - x_{k})^2} \qquad \xi_1 \in (x_k,\sqrt[n]{a})$$
		则有:
		$$
			\lim\limits_{k \rightarrow \infty}\frac{\sqrt[n]{a} - x_{k+1}}{(\sqrt[n]{a} - x_{k})^2} =\frac{\phi_{1}''(\sqrt[n]{a})}{2} = - \frac{n - 1}{2\sqrt[n]{a}}
		$$
		对于$\phi_{2}$可以类似计算出:
				$$
		\lim\limits_{k \rightarrow \infty}\frac{\sqrt[n]{a} - x_{k+1}}{(\sqrt[n]{a} - x_{k})^2} =\frac{\phi_{2}''(\sqrt[n]{a})}{2} = -\frac{n+1}{2\sqrt[n]{a}}
		$$\\
		\\
		(3)我们只需要调整系数消去二次项即可,$c_1$与$c_2$需要满足以下方程:
		$$
			\left\{
				\begin{array}{lcl}
				c_1 + c_2 = 1 \\
				\phi_{1}''(\sqrt[n]{a})c_1 + \phi_{2}''(\sqrt[n]{a})c_2 = 0
				\end{array}
			\right.
		$$
		解得:
			$$
		\left\{
		\begin{array}{lcl}
		c_1 = \frac{n - 1}{2}\\
		c_2 = \frac{3 - n}{2}
		\end{array}
		\right.
		$$
		所以$\phi =\frac{n - 1}{2}\phi_1 + \frac{3 - n}{2}\phi_2$\\\\
		\item[4.]设 $a > 0 $, $n$ 为正整数, 给定迭代法
		$$
		x_{k+1} = \frac{x_k(x_k^2 + 3a)}{3x_k^2 + a}
		$$
		证明:\\
		(1)迭代法是计算$\sqrt{a}$的三阶方法\\
		(2) 假定初值$x_0$充分靠近$x_*,$计算$\lim\limits_{k\rightarrow \infty}(\sqrt{a} - x_{k+1})/(\sqrt{a} - x_k)^3.$ \\
		\\
		(1)记$\phi(x) = \frac{x(x^2 + 3a)}{3x^2 + a}$,求其三阶导数:
		$$\phi'(x) = 3\left(\frac{x^2 - a}{3x^2 + a}\right)^2$$
		$$\phi''(x) = \frac{48ax(x^2 - a)}{(3x^2 + a)^3}$$
		$$\phi'''(x) = \frac{48(9ax^4 - a^3 - 18x^4 + 18a^2x^2)}{(3x^2 + a)^4}$$
		可以验证$\phi'(\sqrt{a}) = 0\,,\phi''(\sqrt{a}) = 0\,,\phi'''(\sqrt{a}) = \frac{3}{2a}$
		由此可以看出这是计算$\sqrt(a)$的三阶方法。\\\\
		(2)由题意得:
		$$\frac{\sqrt{a} - x_{k+1}}{(\sqrt{a} - x_k)^3} = \frac{\phi(\sqrt{a}) -\phi(x_{k})}{(\sqrt{a} - x_k)^3}= \frac{\phi'''(\xi)}{3!} \qquad \xi \in (x_k,\sqrt{a}) $$
		$$\lim\limits_{k \rightarrow \infty}\frac{\sqrt{a} - x_{k+1}}{(\sqrt{a} - x_{k})^3} = \frac{\phi'''(\sqrt{a})}{3!} = \frac{1}{4a}$$\\
		\item[5.]考虑方程 $3x^2 − e^x = 0. $给定容许误差 $\epsilon = 10^{−6}$\\
		(1) 分析方程根的分布情况.\\
		(2) 对于每个根,构造不动点算法求根,利用 MATLAB 编程求解,记录迭代次数并观
		察给定不同初值的结果.
\\
		(3) 对于每个根,构造 Newton 法求根,利用 MATLAB 编程求解,记录迭代次数并观
		察给定不同初值的结果.\\\\
		
		(1)利用MATLAB编程,分析解所在的大致区域:
		\begin{lstlisting}
			function analyze()
			x  = -10:0.01:10;
			res = 3*x.^2 - exp(x);
				for i = 1:length(res) - 1
					if res(i) * res(i+1) < 0
						fprintf("(%.2f, %.2f)\n",x(i), x(i + 1));
					end
				end
			end
		\end{lstlisting}
		得到$(-0.46, -0.45)\,,(0.91, 0.92)\,,(3.73, 3.74)$,确定解所在的大致区间。\\
		\\
		(2)根据题意可以构造\\
		对于在$(-0.46, -0.45)$的根,构造得到函数:
		$$\phi_1(x) = \frac{3x^2 - e^x + 2x}{2} = x$$
		求导数得到:
		$$\phi_1'(x) = \frac{6x - e^x +2}{2} \le \phi_{1}'(-0.45) = L_1$$
		编写MATLAB程序:
		\begin{lstlisting}
			function root1(st)
			eps = 1e-6;
			cnt = 0;
			xk1 = 0;
			xk = st;
			L =  (6*(-0.45)-exp(-0.45)+ 2)/2;
			while(1)
				cnt = cnt + 1;
				xk1 = (3 * xk.^2 - exp(xk) +2*xk)/2;
				if(abs(xk1 - xk)/(1 - L) <= eps)
					break;
				end
				xk = xk1;
			end
			fprintf("result is %.7f, iterate %d times\n", xk, cnt);
			end
		\end{lstlisting}
		输入-0.45可以得到结果:result is -0.4589632, iterate 26 times.并且离精确解越近,迭代次数越少。\\\\
		
		对于在$(0.91, 0.92)$的根,构造得到函数:
		$$\frac{e^x}{3x} = x$$
		求导可以得到Lipschitz系数:
		$$|\phi_{2}'(x)| =\left| e^{x}\left(\frac{x- 1}{3x^2}\right)\right| \le |\phi_{2}'(0.91)| = L_2$$
		则可以利用此结果编写MATLAB程序:
		\begin{lstlisting}
			function root2(st)
			eps = 1e-6;
			cnt = 0;
			xk1 = 0;
			xk = st;
			L =  exp(0.91).*(0.91 - 1) / (3*0.91.^2);
			while(1)
				cnt = cnt + 1;
				xk1 = exp(xk)./(3.*xk);
				if(abs(xk1 - xk)/(1 - L) <= eps)
					break;
				end
				xk = xk1;
			end
			fprintf("result is %.7f, iterate %d times\n", xk, cnt);
			end
		\end{lstlisting}
		得到结果:result is 0.9100082, iterate 5 times,改变起始节点的效果与$(1)$类似。\\\\

		对于在$(3.73, 3.74)$的根,构造得到函数:
		$$\frac{3x^3}{e^x} = x$$
		求导可以得到Lipschitz系数:
		$$|\phi_{3}'(x)| =\left|\frac{9x^2 - 3x^3}{e^{2x}}\right| \le |\phi_{3}'(3.74)| = L_2$$
		则可以利用此结果编写MATLAB程序:
		\begin{lstlisting}
		function root3(st)
		eps = 1e-6;
		cnt = 0;
		xk1 = 0;
		xk = st;
		L =  (9*3.74^2 - 3*3.74^3) / exp(3.74);
		while(1)
			cnt = cnt + 1;
			xk1 = 3*xk.^3./exp(xk);
			if(abs(xk1 - xk)/(1 - L) <= eps)
				break;
			end
			xk = xk1;
		end
		fprintf("result is %.7f, iterate %d times\n", xk, cnt);
		end
		\end{lstlisting}
		输入3.73,得到结果:result is 3.7330781, iterate 27 times,改变起始节点的效果与$(1)$类似。\\\\
		(2)计算牛顿公式得到:
		$$x_{k+1} = x_k - \frac{3x_k^2 - e^x_k}{6x_k - e^x_k}$$
		编写MATLAB程序:
		\begin{lstlisting}
			function Newton(st)
			eps = 1e-6;
			cnt = 0;
			xk1 = 0;
			xk = st;
			while(1)
				cnt = cnt + 1;
				xk1 = xk - (3*xk.^2 - exp(xk))./(6.*xk - exp(xk));
				if(abs(xk1 - xk)/(xk) < eps)
					break;
				end
				xk = xk1;
			end
			fprintf("result is %.7f, iterate %d times\n", xk, cnt);
			end
		\end{lstlisting}
		分别输入$-1,\, 1,\, 3$,得到:\\
		result is -0.4589623, iterate 5 times\\
		result is 0.9100076, iterate 4 times\\
		result is 3.7330791, iterate 9 times\\
		速度远好于不动点迭代法。
	\end{itemize}
\end{document}