\documentclass{article}
\usepackage{ctex}
\usepackage{geometry}
\usepackage{amsfonts,amssymb, amsmath}
\usepackage{listings} 
\usepackage{fontspec}
\usepackage{graphicx}
\geometry{a4paper,left=2.5cm,right=2.5cm,top=2cm,bottom=2cm}
\newcommand*{\dif}{\mathop{}\!\mathrm{d}}
\renewcommand{\baselinestretch}{1.5}
\title{科学计算作业8}	
\author{张文涛 517030910425}
\date{\today}
\lstset{columns=flexible,numbers=left,numberstyle=\tiny,basicstyle=\small,keywordstyle=\color{blue!70},commentstyle=\color{red!50!green!50!blue!50}, rulesepcolor= \color{ red!20!green!20!blue!20} }
\begin{document}
	\maketitle
	\begin{itemize}
		\item[1.]假设函数$f(x)$充分光滑,$h > 0$且$h$很小,确定下列数值微分公式的误差.
		\begin{align*}
		&(1)f'(x_0)\approx \frac{1}{4h}[f(x_0 + 3h)-f(x_0 - h)];\\
		&(2)f'(x_0)\approx \frac{1}{2h}[4f(x_0 + h) - 3f(x_0) - f(x_0 + 2h)].
		\end{align*}\\\\
		(1)利用$f(x)$在$x_0$处的Taylor展开:
		$$f(x) = f(x_0) + f'(x_0)(x - x_0) + \frac{f''(x_0)}{2!}(x - x_0)^2 + \frac{f'''(x_0)}{3!}(x - x_0)^3+\cdots$$
		计算$f(x_0 + 3h),f(x_0 - h)$,得到:
		$$
			f(x_0 + 3h) =  f(x_0) + 3hf'(x_0) + \frac{9h^2}{2}f''(\xi_1)
		$$
		$$
			f(x_0 -h) = f(x_0) - hf'(x_0) + \frac{h^2}{2}f''(\xi_1)
		$$
		则可以求得:
		$$
			 \frac{1}{4h}[f(x_0 + 3h)-f(x_0 - h)] = f'(x_0) + hf''(\xi_1)
		$$
		所以误差为$O(h)$的,且不超过$h||f''(x)||$。\\\\
		(2)利用类似的方法得到:
		$$
			f(x_0 + h)  =  f(x_0) + hf'(x_0) + \frac{h^2}{2}f''(x_0) + \frac{h^3}{3!}f'''(\xi_2)
		$$
		$$
			f(x_0 + 2h) = f(x_0 ) + 2hf'(x_0) + \frac{4h^2}{2}f''(x_0) + \frac{8h^3}{3!}f'''(\xi_2)
		$$
		可以类似求得:
		$$
			\frac{1}{2h}[4f(x_0 + h) - 3f(x_0) - f(x_0 + 2h)] = f'(x_0) - \frac{h^2}{3}f'''(\xi_2)
		$$
		所以误差是$O(h^2)$的,且不超过$\frac{h^2}{3}||f'''(x)||$。\\\\
		
		\item[2.]记$C_p[0,2\pi]$中以$2\pi$为周期的函数全体所成集合.设$f(x)\in C_p[0, 2\pi]$,对区间进行等距剖分, 相应节点为
			$$x_j = \frac{2\pi j}{2N}, \quad j = 0,1,\cdots,2N - 1.$$
		给定求积公式
		$$\frac{1}{2\pi}\int_{0}^{2\pi}f(x)\dif x \approx \frac{1}{2N}\sum_{j =  0}^{2N - 1}f(x_j),$$
		证明:当$f(x) =e^{ikx},|k| \le 2N - 1$时,求积公式是精确的.\\\\
		证:计算左边积分:
		$$\frac{1}{2\pi}\int_{0}^{2\pi}e^{ikx}\dif x = \left.\frac{1}{2\pi ik}e^{ikx}\right|_{0}^{2\pi} = 0$$
		右边求和:
		$$ \frac{1}{2N}\sum_{j =  0}^{2N - 1}e^{ikx_j} = \frac{1}{2N}\sum_{j =  0}^{2N - 1}e^{\frac{ik\pi}{N}j} = \frac{ 1 - (e^{\frac{ik\pi}{N}})^{2N}}{1 - e^{\frac{ik\pi}{N}}}$$
		由于$|k| \le 2N - 1$,所以$|\frac{k\pi}{N}|< 2\pi$,则有:
		$$\frac{ 1 - (e^{\frac{ik\pi}{N}})^{2N}}{1 - e^{\frac{ik\pi}{N}}} = 0$$
		故积分值是精确的。\\\\
		\item[3.]证明等式
		$$n\sin \frac{\pi}{n} = \pi - \frac{\pi ^{3}}{3! n^2} + \frac{\pi ^5}{5!n^4} - \cdots,$$
		并根据$n\sin\frac{\pi}{n}(n = 3, 6, 12)$,用Richardson外推法求$\pi$的近似值.\\\\
		证:利用$sin(x)$在$0$处的Taylor展开:
		$$sin(x) = x - \frac{x^3}{3!} + \frac{x^5}{5!} - \frac{x^7}{7!}+\cdots$$
		将$x =  \frac{\pi}{n}$代入,并在两边同乘以$n$,得到:
		$$n\sin \frac{\pi}{n} = \pi - \frac{\pi ^{3}}{3! n^2} + \frac{\pi ^5}{5!n^4} - \cdots$$
		得证。\\
		我们可以精确计算得到:
		$$3\sin\frac{\pi}{3} = \frac{3\sqrt{3}}{2}$$
		$$6\sin\frac{\pi}{6} = 3$$
		$$12\sin \frac{\pi}{12} = 3\sqrt{6} - 3\sqrt{2}$$
		利用Richardson外推法,先利用$n = 3,\, n = 6$进行一次外推,再利用$n = 6,\, n = 12$进行外推,再利用两次外推的结果再进行一次外推。\\
		利用$n=3, 6$的情况进行外推:
		$$\pi - 3\sin\frac{\pi}{3} - 4\pi + 24\sin\frac{\pi}{6} =  (\frac{1}{5!3^4} - \frac{4}{5!6^4})\pi ^5 + (\frac{4}{7!6^6} - \frac{1}{7!3^6})\pi ^7+\cdots$$
		再利用$n = 6, 12$的情况进行外推:
		$$\pi - 6\sin\frac{\pi}{6} - 4\pi + 48\sin\frac{\pi}{12} = (\frac{1}{5!6^4} - \frac{4}{5!12^4})\pi ^5 + (\frac{4}{7!12^6} - \frac{1}{7!6^6})\pi ^7+\cdots$$
		再利用这两个结果进行外推近似计算:
		$$\pi - 3\sin\frac{\pi}{3} - 4\pi + 24\sin\frac{\pi}{6} - 16\left(\pi - 6\sin\frac{\pi}{6} - 4\pi + 48\sin\frac{\pi}{12}\right) = 0$$
		计算得到:
		$$\pi \approx \frac{3\sin\frac{\pi}{3}-120\sin\frac{\pi}{6}+768\sin\frac{\pi}{12}}{45}= 3.141580063$$\\
		\item[4.]假设$f(x) \in C^5[a, b],$对于求积公式
		$$\int_{a}^{b}f(x)\approx\frac{b - a}{30}\left[7f(a) + 16f\left(\frac{a+b}{2}\right) + 7f(b)\right] +\frac{(b - a)^2}{60}\left[f'(a) -f'(b)\right],$$
		(1)试确定其代数精度\\
		(2)给出该求积公式的误差估计.\\
		\\
		(1)经过检验,当$f(x)$取$c,\,x,\,x^2,\,x^3,\,x^4,\,x^5$时,原式精确成立,而将$f(x) = x^6$代入检验后原式不成立,因此求积公式有五次代数精度。\\
		(2)由于求积公式含有导数,因此考虑为满足如下条件的Hermite插值求积,公式:
		$$
			\left\{
			\begin{array}{lcl}
			P_4(a) = f(a)\\
			P_4'(a) = f'(a)\\
			P_4(b) = f(b)\\
			P_4'(b) = f'(b)\\
			P_4(\frac{a+b}{2}) = f(\frac{a+b}{2})
			\end{array}
			\right.
		$$
		构造得到Hermite 多项式,可以验证:
		$$\int_{a}^{b}P_4(x) =\frac{b - a}{30}\left[7f(a) + 16f\left(\frac{a+b}{2}\right) + 7f(b)\right]$$
		则Hermite插值的余项可以写为:
		$$
			R_n(x) = \frac{f^{(5)}(\xi(x))}{5!}(x-a)^2(x-b)^2(x-\frac{a+b}{2})
		$$
		所以余项为:
		$$
			\int_{a}^{b}\frac{f^{(5)}(\xi(x))}{5!}(x-a)^2(x-b)^2(x-\frac{a+b}{2})
		$$\\
		
		\item[5.]基于梯形求积公式,导出相应的自适应数值积分方法.\\
		\\
		已经推导出区间$[a, b]$梯形求积公式的余项公式
		$$I - S_1 = -\frac{(b-a)^3}{12}f''(\eta)$$
		对区间二分利用梯形公式求积,则有
		$$I - S_2 = -\frac{(b-a)^3}{48}f''(\xi)$$
		需要假设在积分区间内$f(x)$的二阶导数值变化不大,因此有$f''(\xi) = f''(\eta)$,利用这一点:
		$$\left|S_2 - S_1\right| = \frac{3}{4}\left|\frac{(b-a)^3}{12}f''(\eta)\right| = 3\left|I - S_2\right|$$
		若规定误差为$\epsilon$,则当在区间$[a,b]$上$|S_2 - S_1| < 3\epsilon$时,可以利用$S_2$作为该区间上的近似。\\
		\\
		
		\item[6.]已知
		$$I = \int_{0}^{1}\frac{4}{1 + x^2}\dif x = \pi$$
		利用MATLAB编程实现Romberg算法,数值计算$\int_{0}^{1}\frac{4}{1 + x^2}\dif x$并完成下列表格.\\
		\\
		利用推到得到的公式计算,编写程序得:
		\begin{lstlisting}
		function arr = Romberg()
		arr = zeros(6, 6);
		for i = 1:6
		pos = 0:1/(2.^(i-1)):1;
		res = zeros(1, length(pos) - 1);
		for j = 1:length(pos) - 1
		res(j) = ((pos(j+1) - pos(j)) / 2 ).* (4/(1+pos(j).^2 ) + 4/(1 + pos(j+1).^2));
		end
		arr(i, 1) = sum(res);
		end
		for i=1:6
		for k = 2:i
		arr(i, k) = (4.^(k-1)).*arr(i, k - 1)/(4.^(k - 1) - 1) - arr(i - 1, k - 1)/(4.^(k - 1)  - 1);
		end
		end
	\end{lstlisting}
		得到结果:\\\\
		
		 \resizebox{\textwidth}{20mm}{
		\begin{tabular}{ccccccc}
			\hline
			$k$ & $T_{0}^{(k)}$  & $T_{1}^{(k)}$ & $T_{2}^{(k)}$ & $T_{3}^{(k)}$ & $T_{4}^{(k)}$ & $T_{5}^{(k)}$\\
			\hline
			
			0& 3.000000000000000&                   - &                 - &                 - &                  -   &                -\\
			1&3.100000000000000  & 3.133333333333334  &                 - &                  - &                  -   &                -\\
			2&3.131176470588236  & 3.141568627450981  & 3.142117647058824 &                  -  &                 -   &                -\\
			3&3.138988494491089  & 3.141592502458707  & 3.141594094125889  & 3.141585783761874  &                 -   &                -\\
			4&3.140941612041389 &  3.141592651224823  & 3.141592661142564 &  3.141592638396796 &  3.141592665277718    &               -\\
			5&3.141429893174974 &  3.141592653552835   &3.141592653708036  & 3.141592653590028 &  3.141592653649609 &  3.141592653638242\\
			\hline
		\end{tabular}
	}
	\end{itemize}
\end{document}