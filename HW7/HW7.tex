\documentclass{article}
\usepackage{ctex}
\usepackage{geometry}
\usepackage{amsfonts,amssymb, amsmath}
\usepackage{listings} 
\usepackage{fontspec}
\usepackage{graphicx}
\geometry{a4paper,left=2.5cm,right=2.5cm,top=2cm,bottom=2cm}
\newcommand*{\dif}{\mathop{}\!\mathrm{d}}
\renewcommand{\baselinestretch}{1.5}
\title{科学计算作业7}	
\author{张文涛 517030910425}
\date{\today}
\lstset{columns=flexible,numbers=left,numberstyle=\tiny,basicstyle=\small,keywordstyle=\color{blue!70},commentstyle=\color{red!50!green!50!blue!50}, rulesepcolor= \color{ red!20!green!20!blue!20} }
\begin{document}
	\maketitle
	\begin{itemize}
		\item[1.]确定下列求积公式中的特定参数,使其代数精度尽量高,并指明其代数精度\\
		$$(1)\int_{-h}^{h}f(x)\,dx\approx Af(-h) + Bf(0) + Cf(h) +Dhf'(h);$$
		$$(2)\int_{0}^{h}f(x)\,dx\approx\frac{h}{2}[f(0) + f(h)] + ah^{2}[f'(0) - f'(h)];$$
		$$(3)\int_{0}^{1}f(x)\,dx\approx Af(0) + Bf(1) + Cf''(0) +Df''(1)$$
		$$(4)\int_{-1}^{1}f(x)\,dx\approx\frac{1}{3}[f(-1)+2f(x_1)+3f(x_2)]$$
		$$(5)\int_{0}^{1}\frac{1}{\sqrt{x}}f(x)\,dx\approx \omega_{0}f(x_{0})+\omega_{1}f(x_1)$$\\
		(1)有四个自由度,可以有三次代数精度,分别将$\mbox{常数},x,x^2, x^3$代入计算得线性方程组:
		\begin{equation}
			\nonumber
			\begin{bmatrix} 1 & 1 & 1 & 0\\-h & 0 & h & h\\h^2 & 0 & h^2 & 2h^2 \\-h^3 & 0 & h^3 & 3h^3\end{bmatrix}
			\begin{bmatrix}A\\B\\C\\D\end{bmatrix}
			=
			\begin{bmatrix}2h \\ 0 \\  \frac{2}{3} h^3 \\ 0 \end{bmatrix}
		\end{equation}
		解得:
		\begin{equation}
			\nonumber
			\begin{bmatrix}A\\B\\C\\D\end{bmatrix}
			=
			\begin{bmatrix}\frac{1}{3}h\\\frac{4}{3}h\\\frac{1}{3}h\\0\end{bmatrix}
		\end{equation}
		经过验证,有三次代数精度。\\\\
		(2)待定系数只有$a$,可以验证当$f(x)$为常数或$x$时精确成立,将$x^2$代入可以计算得到$a = \frac{1}{12}$,并且经过检验对于$x^3$也精确成立。\\\\
		(3)用与(1)类似的方式列出线性方程组:
		\begin{equation}
			\nonumber
			\begin{bmatrix}1 & 1 & 0 & 0\\0 & 1 & 0 & 0 \\ 0 &1 & 2 & 2\\0 & 1 & 0 & 1\end{bmatrix}
			\begin{bmatrix}A\\B\\C\\D\end{bmatrix}
			=
			\begin{bmatrix}1\\\frac{1}{2}\\\frac{1}{3}\\\frac{1}{4}\end{bmatrix}
		\end{equation}
			解得:
		\begin{equation}
		\nonumber
		\begin{bmatrix}A\\B\\C\\D\end{bmatrix}
		=
		\begin{bmatrix}\frac{1}{2}\\\frac{1}{2}\\\frac{1}{6}\\-\frac{1}{4}\end{bmatrix}
		\end{equation}
		经过验证,有三次代数精度。\\\\
		(4)可以验证,对于任意常数,原始精确成立。可以将$f(x) = x $及$f(x) = x^2$代入计算待定项。于是得到方程组:
		$$
		\left\{
			\begin{array}{lcl}
			2x_1 + 3x_2 = 1 \\
			2x_1^2 + 3x_2^2 = 1 
			\end{array}
		\right.
		$$
		解得
		\begin{equation}
		\nonumber
		\left\{
		\begin{array}{lcl}
			x_1 = \frac{1+\sqrt{6}}{5} \\
			x_2 = \frac{3-2\sqrt{6}}{15} 
		\end{array}
		\right.
		\qquad
		\mbox{或}
		\qquad
		\left\{
		\begin{array}{lcl}
		x_1 = \frac{1-\sqrt{6}}{5} \\
		x_2 = \frac{3+2\sqrt{6}}{15} 
		\end{array}
		\right.
		\end{equation}
		经检验$f(x) = x^3$不精确成立,因此有两次代数精度。\\\\
		(5)利用待定系数法,将$f(x) = c, x, x^2, x^3$代入,得到方程组:
		解得两个节点及其对应系数为:
		\begin{equation}
		\nonumber
		\left\{
		\begin{array}{lcl}
		x_0 = \frac{15-2\sqrt{30}}{35} \\
		\omega_{0} = \frac{18+\sqrt{30}}{18} 
		\end{array}
		\right.
		\qquad
		\left\{
		\begin{array}{lcl}
		x_1 = \frac{15+2\sqrt{30}}{35} \\
		\omega_{1} = \frac{18-\sqrt{30}}{18} 
		\end{array}
		\right.
		\end{equation}
		经过验证,有三次代数精度。\\\\
		\item[2.]考虑Newton-Cotes公式
		$$\int_{a}^{b}f(x)\,dx \approx \sum_{i = 0}^{n}A_{i}f(x_i)$$
		式中$x_i = a_i + h,\quad h =(b- a)/n.$\\
		证明:\\
		(1)$A_i = A_{n - i}$\\
		(2)若$n$是偶数,则其具有$n+1$阶代数精度.\\\\
		(1)对原函数作等距插值得到插值多项式:
		$$P_n(x) = \sum_{i = 0}^{n}(\prod_{j = 0\, j\ne i}^{n}\frac{x - x_j}{x_i - x_j})f(x_i)$$
		$$A_{i} = \int_{a}^{b}\prod_{j = 0\, j\ne i}^{n}\frac{x - x_j}{x_i - x_j}\,dx$$
		由于已知$x_{i} = a + ih	$,记$x = a + th,\,t\in [0,1]$则有:
		$$A_{i} = h\int_{0}^{n}\prod_{j = 0\, j\ne i}^{n}\frac{t - j}{i - j}\,dt$$
		$$A_{i} = \frac{(-1)^{n - i}h}{i!(n - i)!} \int_{0}^{n}\prod_{j = 0\, j\ne i}^{n}(t - j)\,dt$$
		如果令$i = n - i$,则有:
		$$A_{n - i} = \frac{(-1)^{i}h}{i!(n - i)!} \int_{0}^{n}\prod_{j = 0\, j\ne n - i}^{n}(t - j)\,dt$$
		令$t = n -t$,则有:
		$$A_{n - i} = \frac{(-1)^{n + i}h}{i!(n - i)!} \int_{0}^{n}\prod_{j = 0\, j\ne n - i}^{n}[t - (n - j)]\,dt$$
		再令$j = n  - j$,则有:
		$$A_{n - i} = \frac{(-1)^{n + i}h}{i!(n - i)!} \int_{0}^{n}\prod_{j = n\, j\ne i}^{0}(t - j)\,dt$$
		由于$(-1)^{(n -i) +(n + i)} = (-1)^{2n} $,即$(-1)^{n+i}$与$(-1)^{n- i}$同号,于是$A_i = A_{n - i}$.\\\\
		(2)假设$f(x)$为$n+1$次多项式,考虑余项,其中$l_i(x)$为Lagrange插值基函数:
		$$R_n(x) = \int_{a}^{b}(f(x) - \sum_{i=0}^{n}l_i(x)f(x_i)) \,dx$$
		则由Lagrange插值余项公式得到:
		$$R_n(x) = \int_{a}^{b}\frac{f^{(n+1)}(\xi)}{(n+1)!}\prod_{i = 0}^{n}(x - x_i)\,dx =\frac{f^{(n+1)}(\xi)}{(n+1)!}\int_{a}^{b}\prod_{i = 0}^{n}(x - x_i)\,dx$$
		利用与(1)类似替换得到:
		$$R_n(x) =h^{n + 2}\frac{f^{(n+1)}(\xi)}{(n+1)!}\int_{0}^{n}\prod_{i = 0}^{n}(t - i)\,dt$$
		令$n = 2k$,有
		$$R_n(x) =h^{2k + 2}\frac{f^{(n+1)}(\xi)}{(n+1)!}\int_{0}^{2k}\prod_{i = 0}^{2k}(t - i)\,dt$$
		令$t = p + k$,代入换元得到:
		$$R_n(x) =h^{2k + 2}\frac{f^{(n+1)}(\xi)}{(n+1)!}\int_{-k}^{k}\prod_{i = 0}^{2k}(t  - (i - k))\,dt = h^{2k + 2}\frac{f^{(n+1)}(\xi)}{(n+1)!}\int_{-k}^{k}t\prod_{i = 0}^{k}(t^2  - i^2)\,dt$$
		$t\prod_{i = 0}^{k}(t^2  - i^2)\,dt$是奇函数,于是$R_n(x) = 0$,有n+1次代数精度。\\\\
		\item[3.]对$f(x), g(x)\in C^{1}[a, b]$,定义
		$$\begin{array}{lcl}
		(1)(f,g)=\int_{a}^{b}f'(x)g'(x)\,dx;\\
		(2)(f,g)=\int_{a}^{b}f'(x)g'(x)\,dx + f(a)g(a);
		\end{array}$$(
		试问上述表示两种定义是否构成内积并说明原因.\\\\
		(1)容易验证,原式满足对称性,线性性,正定性,但有:
		$$\int_{a}^{b}[f'(x)]^2\,dx \not\Rightarrow f(x) = 0$$
		因此不构成内积空间。\\
		
		(2)\\
		2.1显然,满足对称性。\\
		2.2$\,\forall\, V(x) \in C^{1}[a, b]\,$有:
		$$\int_{a}^{b}(f(x) +V(x))'g'(x)\,dx + (f(a) +V(a))g(a) = \int_{a}^{b}f'(x)g'(x)\,dx + f(a)g(a) + \int_{a}^{b}V'(x)g'(x)\,dx +V(a)g(a)$$
		满足线性性。\\
		2.3显然满足正定性\\
		2.4取$f(x) = g(x)$,原式变为:
		$$\int_{a}^{b}[f'(x)]^2\,dx + f^2(a) = 0$$
		可以看出$f(x)$为常数且为$0$,因此构成内积空间。\\\\
		4.试证明$[-1,1]$上Gauss-Legendre求积公式$(\rho(x) = 1)$的节点和求积系数关于原点$x = 0$是对称分布的。\\\\
		证:Gauss-Legendre多项式的零点是可以被确定的,由Legendre多项式的定义:
		$$\phi_k(x)=\frac{1}{2^kk!}\frac{\dif^k}{\dif x^k}[(x^2 - 1)^k]$$
		现需要证明对于$\forall k \in \mathbb{N^{*}}$,若$x_{0}$使$\phi_{k}(x_0) = 0$,则$\phi_{k}(-x_0) = 0$。\\
		当$k = 1$时,有:
		$$\phi_{1}(x) = x$$
		显然成立。
		当$k = 2$时,有:
		$$\phi_{2}(x) = \frac{3x^2 - 1}{2}$$
		$\phi_{2}(x)$为偶函数,结论显然成立。
		现利用归纳假设,假设$k<p$时结论均成立,要证明$k=p$时结论成立,则利用三项递推关系:
		$$\phi_{p}(x) = \frac{2k+1}{k+1}x\phi_{p - 1}(x) - \frac{k}{k+1}\phi_{p - 2}(x)$$
		容易看出$\phi_{p}(x)$也满足这个结论,因此节点关于原点对称分布。\\
		由于求积系数$A_i$满足:
		$$A_i = \int_{a}^{b}l_{i}(x)\,\dif x$$
		其中$l_i(x)$是以Legendre多项式零点为插值节点的原函数的插值基函数。\\
		所以$l_i(x)$满足:
		$$
		\left\{
		\begin{array}{lcl}
			l_i(x_i) = 1\\
			l_j(x_j) = 0 & j \ne i
		\end{array}
		\right.
		$$
		因此可以确定:
		$$A_i = \int_{-1}^{1}\prod_{j = 0\, j\ne i}^{n}\frac{x - x_j}{x_i - x_j}\,\dif x$$
		欲证:$A_i = A_{n -i}$\\
		$$A_{n - i} =  \int_{-1}^{1}\prod_{j = 0\, j\ne n-i}^{n}\frac{x - x_j}{x_{n-i} - x_j}\,\dif x$$
		由于$x_{n -i} = -x_i$,由此得到:
		$$A_{n - i} =  \int_{-1}^{1}\prod_{j = 0\, j\ne n-i}^{n}\frac{x - x_j}{-x_{i} - x_j}\,\dif x$$
		令$j = n -j$,得到:
		$$A_{n - i} =  \int_{-1}^{1}\prod_{j = n\, j\ne i}^{0}\frac{x + x_j}{-x_{i} + x_j}\,\dif x$$
		令$x = -x$,得到:
		$$A_{n - i} =  \int_{-1}^{1}\prod_{j = n\, j\ne i}^{0}\frac{x - x_j}{x_{i} - x_j}\,\dif x$$
		于是得证。\\\\
		\item[5.]考虑加权Gauss求积公式
		$$\int_{a}^{b}\rho(x)f(x)\approx \sum_{j = 0}^{n}A_jf(x_j)$$
		记其误差为
		$$R_n(f)=\int_{a}^{b}\rho(x)f(x)\,\dif x - \sum_{j = 0}^{n}A_jf(x_j).$$
		求证:
		$$\left|R_n(f)\right| \le 2\left(\int_{a}^{b}\rho(x)\,\dif x\right)E_{2n+1}(f),$$
		式中:
		$$E_{2n+1}(f) =\min_{p\in\mathbb{P}_{2n+1}}||f - p||_{\infty}$$\\
		证:容易知道:
		$$\sum_{j = 0}^{n}A_jf(x_j) = \int_{a}^{b}\rho(x)P_n(x)\dif x$$
		其中$P_n(x)$为以$n+1$次带权正交多项式$\omega_{n+1}(x)$零点为插值节点的原函数的插值多项式。\\
		令$P_{2n+1}(x)\rho(x) = \omega_{n+1}Q_n(x)\rho(x) + P_n(x)\rho(x)$,其中$Q_n(x)$为任意$n$次多项式,对两边积分,则有:
		$$\int_{a}^{b}P_{2n+1}(x)\rho(x)\dif x = \int_{a}^{b}P_n(x)\rho(x) \dif x$$
		代入则有:
		$$
			R_n(f) = \int_{a}^{b}\rho(x)(f(x) - P_{2n+1}(x))\,\dif x
		$$
		$$
			\left|R_n(f)\right| \le \int_{a}^{b}\rho(x)\left|(f(x) - P_{2n+1}(x))\right|\,\dif x
		$$
		取合适的$Q_n(x)$,就可以有:
		$$	\left|R_n(f)\right| \le \left|f(\xi) - P_{2n+1}(\xi)\right| \int_{a}^{b}\rho(x)\,\dif x \le 2E_{2n+1}(f) \int_{a}^{b}\rho(x)\,\dif x $$
		得证。\\\\
		\item[6.]已知函数$y = f(x)$在节点 $0 = x_0 < x_1 < \cdots< x_n = 1 $处的近似值 $\tilde{y}_i = f(x_i),
		0 \le i \le n.$ 试用变分方法重构函数$y = f(x)$使其一阶导函数均方取值不大.\\\\
		证:构造泛函
		$$J(y) = \sum_{i = 0}^{n - 1}\frac{h_i +h_{i+1}}{2}(\tilde{y}-y(x_i))^2 +\alpha\int_{0}^{1}(y'(x))^2\dif x$$
		假设$g(x)$为最优解,求其变分:
		$$\delta J = \sum_{i = 0}^{n - 1}(h_i +h_{i+1})(g(x_i) - \tilde{y_i})v(x_i) + 2\alpha\int_{0}^{1}g'(x)v'(x) $$
		做一些转化:
		$$\delta J = \sum_{i = 0}^{n - 1}(h_i +h_{i+1})(g(x_i) - \tilde{y_i})v(x_i) + 2\alpha\sum_{i = 0}^{n - 1}\int_{x_i}^{x_{i+1}}g'(x)v'(x)\dif x $$
		作分部积分:
		$$\delta J = \sum_{i = 0}^{n - 1}(h_i +h_{i+1})(g(x_i) - \tilde{y_i})v(x_i) + 2\alpha\sum_{i = 0}^{n - 1}\left(\left. g'(x)v(x)\right|^{x_{i+1}}_{x_i} - \int_{x_i}^{x_{i+1}}g''(x)v(x)\dif x\right) = 0$$
		令每一部分都为0可以得到解。
		$$\int_{x_i}^{x_{i+1}}g''(x)v(x)\dif x = 0\, \Rightarrow \, g(x) \in \mathbb{P}_{1}$$
		考虑$v(x_i)$的系数等于零:
		$$(h_i + h_{i+1})(g(x_i) - \tilde{y}_{i} + 2\alpha(g'(x_{i} - )+ g'(x_{i}+)) = 0$$
		因此重构的函数为分段一次多项式且满足以上约束条件。\\\\
		
		\item[7.]将积分区间$[0,1]$等距剖分,其格点为${x_i = i/n}^{n}_{i = 0},n = 2, 4, 8, 16, 32.$利用MATLAB使用不同的复化方式在上述网络上计算积分:
		$$I = \int_{0}^{1}e^{-t}dt.$$
		(1)复化梯形公式;\\
		(2)复化Simpson公式;\\
		(3)复化两点Gauss-legendre公式,即$\int_{-1}^{1}f(x)\dif x\approx \omega_{1}f(x_1) + \omega_{2}f(x_2).$\\\\
		利用公式,编写函数:
		\begin{lstlisting}[language=MATLAB]
		function res = intergrate(x)
		p =  length(x);
		res1 = ones(1, p - 1);
		res2 = res1;
		res3 = res1;
		for i = 1:p - 1
		res3(i) = (x(i +1) - x(i))/2 * (exp(1).^(-(x(i + 1) - x(i))/(2*3.^(0.5)) - (x(i+1) + x(i))/ 2) + exp(1).^((x(i+1) -x(i))/ (2*3.^0.5) - (x(i+1) +x(i))/ 2));
		end
		for i = 1:p - 1
		res1(i) = (x(i+1) - x(i))/2 * (exp(-x(i)) + exp(-x(i+1)));
		end
		for i = 1:p - 1
		res2(i) = (x(i+1) - x(i))/6 * (exp(-x(i)) + 4*exp(-(x(i) +x(i+1))/2) + exp(-x(i+1)));
		end
		res = [sum(res1), sum(res2), sum(res3)];
		\end{lstlisting}
		输入等距节点:\\
		在$n = 2\qquad \mbox{时,得到}0.645235190149177\qquad0.632134175320532\qquad0.632111485668374$\\
		在$n = 4\qquad \mbox{时,得到}0.635409429027693\qquad0.632121414604742\qquad0.632119988381843$\\
		在$n = 8\qquad \mbox{时,得到}0.632943418210480\qquad0.632120612389173\qquad0.632120523122588$\\
		在$n = 16\qquad \mbox{时,得到}0.632326313844500\qquad0.632120562177264\qquad0.632120556596104$\\
		在$n = 32\qquad \mbox{时,得到}0.632172000094072\qquad0.632120559037870\qquad0.632120558689016$\\
	\end{itemize}
\end{document}