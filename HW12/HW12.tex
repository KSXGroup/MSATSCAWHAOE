\documentclass{article}
\usepackage{ctex}
\usepackage{geometry}
\usepackage{amsfonts,amssymb, amsmath}
\usepackage{listings} 
\usepackage{fontspec}
\usepackage{graphicx}
\usepackage{bm}
\newcommand{\xc}[1]{\textbf{\emph{#1}}}
\geometry{a4paper,left=2.5cm,right=2.5cm,top=2cm,bottom=2cm}
\renewcommand{\baselinestretch}{1.5}
\title{科学计算作业12}	
\author{张文涛 517030910425}
\date{\today}
\lstset{columns=flexible,numbers=left,numberstyle=\tiny,basicstyle=\small,keywordstyle=\color{blue!70},commentstyle=\color{red!50!green!50!blue!50}, rulesepcolor= \color{ red!20!green!20!blue!20} }
\begin{document}
	\maketitle
	\begin{itemize}
		\item[1.]对于线性方程组$\xc{Ax} = \xc{b}$, 使用迭代法
		$$
			\xc{x}^{(k)} = \xc{x}^{(k)} + \alpha(\xc{Ax}^{(k)} - \xc{b}),\qquad  k = 0,1,\cdots 
		$$
		求解,若
		$$
			\xc{A} = \begin{bmatrix}
				3 & 1 \\
				2 & 2
			\end{bmatrix}
		$$
		(1)试求$\alpha$的取值范围使得迭代法收敛;\\
		(2)若$\alpha$选取适当的值可以使得上述迭代法收敛,试问$\alpha$取何值时上述迭代法收敛速度最快?\\
		\\
		(1)要求$\xc{I} + \alpha\xc{A}$的最大特征值的绝对值(谱半径),利用特征多项式:
		$$
			|(1 - \lambda)\xc{I} + \alpha\xc{A}| = \lambda^2 - \lambda(5\alpha + 2) + 4\alpha^2 + 5\alpha + 1 = 0
		$$
		解得:
		$$
			\lambda = \left|\frac{5\alpha + 2 \pm |3\alpha|}{2}\right|
		$$
		经过验证,当$\alpha >= 0$时,$|\lambda| \ge 1$,舍去。\\
		当$\alpha < 0$时,有条件:
		$$
			4\alpha + 1 > -1 \Rightarrow \alpha > -\frac{1}{2}
		$$
		所以$\alpha \in (-\frac{1}{2}, 0)$时迭代法收敛。\\
		\\
		(2)利用同一个分段函数,最小值在$|4\alpha +1| = |\alpha + 1|$的临界点取到,即$\alpha = -\frac{2}{5}$取到,此时$\lambda = 0.6$。\\
		\\
		\item [2.]设$\xc{A}$是严格对角占优矩阵, 证明求解线性方程组$\xc{Ax} = \xc{b}$ 的 Jacobi 迭代法收敛.\\
		\\
		证:由Jacobi方法,构造迭代方程:
		$$	
			\xc{A} = \xc{D} -\xc{L} -\xc{U}
		$$
		$$
			\xc{x}^{(k+1)} = (\xc{L}+\xc{U})\xc{D}^{-1}\xc{x}^{(k)} + \xc{D}^{-1}b
		$$
		假设$\lambda$为矩阵$\xc{J} = (\xc{L}+\xc{U})\xc{D}^{-1}$的特征值,考虑$J$的特征多项式:
		$$
			|\xc{I}\lambda - (\xc{L}+\xc{U})\xc{D}^{-1}| = |\xc{D}^{-1}||\xc{D} - (\xc{L} + \xc{U})| = 0
		$$
		由于$|\xc{D}^{-1}| \ne 0$,所以$|\lambda\xc{D} - (\xc{L} + \xc{U})| = 0$,即有:
		$$
			\left|
			\begin{array}{cccc}
			\lambda a_{11} & a_{12} & \cdots & a_{1n}\\
			a_{21} & \lambda a_{22} & \cdots & a_{2n}\\
			\vdots & \vdots & \quad & \vdots\\
			a_{n1} & a_{n2} & \cdots & \lambda a_{nn}\\
			\end{array}
			\right| = 0
		$$
		但当$|\lambda| \ge 1$时,矩阵$\lambda\xc{D} - (\xc{L} + \xc{U})$严格对角占优,因此是非奇异的,所以行列式$|\lambda\xc{D} - (\xc{L} + \xc{U})|\ne 0$,所以可以认为$|\lambda| < 1$,所以$\rho(\xc{D} - (\xc{L} + \xc{U})) < 1$,因此迭代法收敛。\\
		\\
		\item[3.]若$\xc{A}$是严格对角占优的对称矩阵或不可约对角占优的对称矩阵, 且对角线元素为正, 则$\xc{A}$正定.\\
		\\
		证:由于$\xc{A}$为实对称阵,则$\xc{A}$有实特征值,若$\xc{A}$非正定,则存在某一特征值$\lambda_p \le 0$,特征值满足特征多项式:
		$$
			|\xc{A} - \lambda_p\xc{I}| = 0
		$$
		显然若$\lambda_p \le 0$,则$\xc{A} - \lambda_p\xc{I}$是严格对角占优或不可约对角占优的,所以不可能为奇异的,所以$|\xc{A} - \lambda_p\xc{I}| \ne 0$,矛盾。故$\xc{A}$的特征值均大于零,所以为正定的。\\
		\\
		\item[4.]设$\xc{A}\in \mathbb{R}^{n\times n},$ 且$ \xc{A} $对称, 可按照如下步骤研究 Jacobi 迭代与 Gauss-Seidel 迭代求解线性方程组 $\xc{Ax} = \xc{b}$的收敛性.\\
		(1)设$\xc{A}$对称正定,记
		$$
			\|\xc{x}\|_{\xc{A}} = \sqrt{(\xc{Ax},\xc{x})},\quad \in \mathbb{R}^{n}
		$$
		证明$\|\xc{x}\|_{\xc{A}}$是$\mathbb{R}_n$上的一种范数;\\
		(2)设$\xc{A}$对称正定,若存在可逆矩阵$\xc{P}$使得$2\xc{P} - \xc{A}$正定,则迭代法
		$$
			\xc{x}^{(k+1)} =\xc{x} ^{(k)} + \xc{P}^{-1}(b - \xc{Ax}^{(k)}),\quad k = 0, 1,\cdots
		$$
		收敛;\\
		(3)若$\xc{A}$对称, 利用上述得到的结论, 分别导出 Jacobi 迭代与 Gauss-Seidel 迭代求
		解线性方程组 $\xc{Ax} = \xc{b}$ 收敛的充分条件.\\\\
		
		(1)\\
		对于任意$\xc{x}\in \mathbb{R}^{n} \ne \textbf{0}, \,\, \|\xc{x}\|_{\xc{A}} = \sqrt{\xc{x}^{T}\xc{A}\xc{x}}$,由于$\xc{A}$为正定矩阵,所以$\xc{x}^{T}\xc{A}\xc{x} > 0$,所以$\forall x\in \mathbb{R}^{n}, \,\, \|\xc{x}\|_{\xc{A}} \ge 0$且满足$\xc{x} = \textbf{0}$时,$\|\xc{x}\|_{\xc{A}} = 0$,满足正定性。\\
		并且$\forall k \in \mathbb{R},\,\,\|k\xc{x}\|_{\xc{A}} = \sqrt{k^2\xc{x}^{T}\xc{A}\xc{x}} = |k|\sqrt{\xc{x}^{T}\xc{A}\xc{x}} = |k|\|k\xc{x}\|_{\xc{A}}$,满足齐次性\\
		作为范数,还需要满足三角不等式,即:
		$$
			\|\xc{u} + \xc{v}\| \le \|\xc{u}\| + \|\xc{v}\|
		$$
		即要证:
		\begin{align*}
			\sqrt{(\xc{u} +\xc{v})^{T}\xc{A}(\xc{u} + \xc{v})} \le \sqrt{\xc{u}^{T}\xc{A}\xc{u}}+\sqrt{\xc{v}^{T}\xc{A}\xc{v}}\\
			\Leftrightarrow (\xc{u}^{T}\xc{A}\xc{v} + \xc{v}^{T}\xc{A}\xc{u})^2 \le 4(\xc{u}^T\xc{A}\xc{u})(\xc{v}^{T}\xc{A}\xc{v})
		\end{align*}
		由于$\xc{A}$为对称正定,则$\exists\xc{B}$使得$\xc{A} = \xc{B}^{T}\xc{B}$,所以有:
		\begin{align*}
			4(\xc{Bu}, \xc{Bv})^2 \le 4(\xc{Bu},\xc{Bu})(\xc{Bv},\xc{Bv})
		\end{align*}
		即柯西不等式,于是满足三角不等式。于是$\|\xc{x}\|_{\xc{A}} $是范数。\\
		
		(2)考虑$\|\xc{x}\|_A^2$的单调性。\\
		由原式可以推出:
		$$
			\xc{e}^{(k+1)} = (\xc{I} - \xc{P}^{-1}\xc{A})\xc{e}^{(k)}
		$$
		两边减去$\xc{e}^{(k)}$,记$\xc{e}^{(k)} - \xc{e}^{(k + 1)} = \bm{\delta}^{(k)}$,得到:
		$$
				\bm{\delta}^{(k)} = -\xc{P}^{-1}\xc{A}\xc{e}^{(k)}
		$$
		则可以将$(\xc{e}^{(k+1)}, \xc{A}\xc{e}^{(k+1)})$转化表示:
		$$
			(\xc{e}^{(k+1)}, \xc{A}\xc{e}^{(k+1)}) = (\xc{e}^{(k)}, \xc{A}\xc{e}^{(k)}) - (\xc{e}^{(k)}, \xc{A}\xc{P}^{-1}\xc{A}\xc{e}^{k}) - (\xc{P}^{-1}\xc{A}\xc{e}^k, \xc{A}\xc{e}^k) + (\xc{P}^{-1}\xc{A}\xc{e}^{k}, \xc{A}\xc{P}^{-1}\xc{A}\xc{e}^k)
		$$
		将$	\bm{\delta}^{(k)} = -\xc{P}^{-1}\xc{A}\xc{e}^{(k)}$代入得到:
		$$
			(\xc{e}^{(k+1)}, \xc{A}\xc{e}^{(k+1)}) -  (\xc{e}^{(k)}, \xc{A}\xc{e}^{(k)}) = (\bm{\delta}^{(k)}, \xc{A}\bm{\delta}^{(k)}) - 2(\bm{\delta}^{(k)}, \xc{P}\bm{\delta}^{(k)}) 
		$$
		由于$2\xc{P} - \xc{A}$正定,则可看出$(\xc{e}^{(k+1)}, \xc{A}\xc{e}^{(k+1)}) -  (\xc{e}^{(k)}, \xc{A}\xc{e}^{(k)}) \le 0$,所以$\|\xc{x}\|_A^2$单调减,且$\|\xc{x}\|_A^2 \ge 0$,则单调递减有下界。所以$\|\xc{x}\|_A^2$收敛。假设存在极限$\xc{e}_{*}$,代入$\xc{e}^{(k)}$的递推关系得到:
		$$
			\xc{P}^{-1}\xc{A}\xc{e}_{*} = \bm{0}
		$$
		由于$\xc{P}^{-1}\xc{A}$非奇异,所以$\xc{e}_{*}$等于$\bm{0}$,所以迭代法收敛。\\
		\\
		(3)由上面得出的结论,分析Jacobi迭代方法以及GS迭代法的矩阵表达形式:
		$$
			\xc{x}^{(k+1)} = \xc{D}^{-1}(\xc{L}+\xc{U})\xc{x}^{(k)} + \xc{D}^{-1}b = (\xc{I} -  \xc{D}^{-1}\xc{A})\xc{x}^{(k)} + \xc{D}^{-1}\xc{b} \qquad (Jacobi)
		$$
		$$
			\xc{x}^{(k+1)} = (\xc{I} - (\xc{D}-\xc{L})^{-1}\xc{A})\xc{x}^{(k)} + (\xc{D} -\xc{L})^{-1}\xc{b} \qquad (GS)
		$$
		可以看出我们只需要保证$\xc{A}$正定,并分别保证$2\xc{D} - \xc{A}$以及$2(\xc{D}- \xc{L}) - \xc{A}$正定即可。\\
		\\
		
		\item[3.]对$n = 10, 20, 40$, 五对角矩阵
		$$
			\xc{A} = \begin{bmatrix}
			-20 & -8 & 1 & \quad & \quad & \quad & \quad \\
			-8 & -20 & -8 & 1 & \quad & \quad & \quad \\
			1 &-8 & -20 & -8 & 1 & \quad & \quad \\
			\quad & \ddots &  \ddots &  \ddots &  \ddots & \ddots &\quad \\
			 \quad & \quad &1 &-8 & -20 & -8 & 1\\
			 \quad & \quad & \quad & 1 & -8 & 20 & -8\\
			 \quad & \quad & \quad & \quad & 1 & -8 & 20\\
			\end{bmatrix} \in \mathbb{R}^{n\times n}
		$$
		分别用 Jacobi 迭代与 Gauss-Seidel 迭代求解线性方程组 $\xc{Ax} = \textbf{0}$. 取初值 $\xc{x}_0 =(1, 1, · · · , 1)^T \in \mathbb{R}_n,$ 迭代至 $\|\xc{x}^{(k)} - \xc{x}_*\|_{\infty}\le10^{-8}$止, 试比较两种方法的迭代次数和收敛速度.\\
		\\
		显然$\xc{A}$是可逆的,所以即比较两种迭代方式趋近于$\textbf{0}$的速度,首先编写Jacobi迭代法的代码:
		\begin{lstlisting}
		function res = Jacobi(n)
		L =-(diag(ones(1, n - 2), -2) + diag(-8 .* ones(1, n - 1), -1));
		U = L';
		D = diag(ones(1, n).* 20);
		x = ones(1, n)';
		x1 = x;
		eps = 1e-8;
		cnt = 0;
		b = zeros(1,n)';
		q = norm( D\(L + U), inf);
		while (1)
		x1 =  D\((L + U) * x +  b);
		cnt = cnt + 1;
		q1 = norm(x1 - x, inf);
		if(abs(q/ (1-q) * q1) < eps) 
		break;
		end
		x = x1;
		end
		res = x;
		fprintf("iteration times: %d\n",cnt);
		end
		\end{lstlisting}
		在输入为$n = 10, 20, 40$时分别得到结果:\\
		iteration times: 58 \\
		iteration times: 106 \\
		iteration times: 133\\
		作为对比,编写GS迭代法的代码:
		\begin{lstlisting}
		function res = Gauss(n)
		L =-(diag(ones(1, n - 2), -2) + diag(-8 .* ones(1, n - 1), -1));
		U = L';
		D = diag(ones(1, n).* 20);
		x = ones(1, n)';
		x1 = x;
		eps = 1e-8;
		cnt = 0;
		b = zeros(1,n)';
		q = norm((D - L) \ U, inf);
		while(1)
		x1 = (D - L) \ (U * x + b);
		q1 = norm(x1 - x, inf);
		cnt = cnt + 1;
		if(abs(q/(1-q) * q1 )< eps)
		break;
		end
		x = x1;
		end
		fprintf("iteration times: %d\n", cnt);
		res = x1;
		end
		\end{lstlisting}
		同样分别输入$n = 10, 20, 40$,得到结果:\\
		iteration times: 28\\
		iteration times: 31\\
		iteration times: 31\\
		相比较而言显然GS迭代法收敛速度更快。
	\end{itemize}
\end{document}