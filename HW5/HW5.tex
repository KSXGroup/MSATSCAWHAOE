\documentclass{article}
\usepackage{ctex}
\usepackage{geometry}
\usepackage{amsfonts,amssymb}
\usepackage{listings} 
\usepackage{fontspec}
\usepackage{graphicx}
\geometry{a4paper,left=2.5cm,right=2.5cm,top=2cm,bottom=2cm}
\renewcommand{\baselinestretch}{1.5}
\title{科学计算作业5}	
\author{张文涛 517030910425}
\date{\today}
\lstset{columns=flexible,numbers=left,numberstyle=\tiny,basicstyle=\small,keywordstyle=\color{blue!70},commentstyle=\color{red!50!green!50!blue!50}, rulesepcolor= \color{ red!20!green!20!blue!20} }
\begin{document}
	\maketitle
	\begin{itemize}
		\item[1.]对权函数$\rho(x) = 1 + x^{2}$,区间$[-1,1]$,试求首项系数为$1$的正交多项式$\phi_{n}(x)$,$n = 0,1,2,3$\\
		利用正交化过程求解。令$\phi_{0} = 1$,利用正交化过程,有:
		$$
			\phi_{1} =x - \frac{\int_{-1}^{1} (1 + x^2)x\,dx}{\int_{-1}^{1}1+x^2\,dx} = x
		$$
		$$
			\phi_{2} =x^2 - \frac{\int_{-1}^{1} (1 + x^2)x^2\,dx}{\int_{-1}^{1}1+x^2\,dx}  - \frac{\int_{-1}^{1} (1 + x^2)x^3\,dx}{\int_{-1}^{1}x^{2}(1+x^2)}x\,dx= x^2 - \frac{2}{5}
		$$
		$$
			\phi_{3} =x^3 - \frac{\int_{-1}^{1} (1 + x^2)x^3\,dx}{\int_{-1}^{1}1+x^2\,dx}  - \frac{\int_{-1}^{1} (1 + x^2)x^4\,dx}{\int_{-1}^{1}x^{2}(1+x^2)\,dx}x -\frac{\int_{-1}^{1}(1+x^2)(x^2-\frac{2}{5})x^{3}\,dx}{\int_{-1}^{1}(1+x^2)(x^2 - \frac{2}{5})^{2}\,dx}(x^2 - \frac{2}{5}) = x^3 - \frac{9}{14}x
		$$
		所以得到:
		$$
		\left\{
			\begin{array}{lcl}
				\phi_{0} = 1, \\
				\phi_{1} =x,\\
				\phi_{2} =x^2 -  \frac{2}{5}\\
				\phi_{3} =  x^3 - \frac{9}{14}x\\
			\end{array}
		\right.
		$$\\
		\item[2.]令$s_{n} = \frac{1}{n+1}T_{n+1}'(x), n \ge 0$.$s_{n}$称为第二类Chbeyshev多项式,试求出$s_{n}$的表达式并证明${s_{n}}$是$[-1,1]$上带权函数$\rho(x) = \sqrt{1-x^{2}}$的正交多项式序列.\\
		$$
		\frac{T_{n+1}'(x)}{n+1} = \int_{-1}^{1}\frac{\sin[(n+1)\arccos x]}{\sqrt{1-x^2}}\, dx
		$$
		任取$p, q \in [-1, 1]$,
		$$
			\int_{-1}^{1}\rho(x)s_{p}(x)s_{q}\,dx = \frac{\sin[(p+1)\arccos x] \sin[(q+1)\arccos x]}{\sqrt{1-x^2}}
		$$
		令$t = \arccos x$,作变量代换,改变积分上下限:
		$$
				\int_{-1}^{1}\rho(x)s_{p}(x)s_{q}\,dx =\int_{2\pi}^{0} -\sin[(p+1)t] \sin[(q+1)t]\,dt
		$$
		则有:
		$$
		\int_{-1}^{1}\rho(x)s_{p}(x)s_{q}(x)\,dx = \left\{
			\begin{array}{lcl}
			0 && p\ne q,\\
			\pi && p = q
			\end{array}
		\right.$$
		所以${s_{n}}$是$[-1,1]$上带权函数$\rho(x) = \sqrt{1-x^{2}}$的正交多项式序列.\\\\
		\item[3.]设$n$为非负整数,$T_{n}(x)$为Chebyshev多项式,计算
		$$
			\int_{-1}^{1} \frac{[T_{n}(x)]^2}{\sqrt{1 - x^2}}dx
		$$
		由题意可知:
		$$
		\int_{-1}^{1} \frac{[T_{n}(x)]^2}{\sqrt{1 - x^2}}dx = \int_{-1}^{1}\frac{\cos^2(n \arccos(x))}{\sqrt{1 - x^2}}dx 
		$$
		令$t = \arccos x$,作变量代换,改变积分上下限:
		$$
		\int_{-1}^{1} \frac{[T_{n}(x)]^2}{\sqrt{1 - x^2}}dx = \int_{0}^{2\pi}\cos^2(nt)dt = \int_{0}^{2\pi}\frac{\cos(2nt) + 1}{2}
		$$
		得到结果:
		$$
		\int_{-1}^{1} \frac{[T_{n}(x)]^2}{\sqrt{1 - x^2}}dx = \left\{
		\begin{array}{lcl}
		\pi && n \ne 0,\\
		2\pi && n = 0
		\end{array}
		\right.
		$$\\\\
		\item[4.]设$x_{0}, x_{1}, \cdots, x_{n}$为区间 $[a, b]$上相异节点,$ω_{n+1}(x) = \prod_{i = 0}^{n}(x − x_{i})$, 证明:

		$$\max\limits_{x_{0}\le x \le x_{n}}|\omega_{n+1}(x)|\le\frac{1}{4}h^{n+1}n!$$
		式中$h = \max\limits_{0 \le i \le n+1}(x_{i+1} - x_{i})$\\\\
		假设$h = x_{k+1} - x_{k}$,则由基本不等式有当$x = \frac{x_{k+1} + x_{k}}{2}$时:
		$$|(x - x_{k})(x - x_{k+1})| \le  \frac{1}{4}h^{2}$$
		另外有,当某个确定的$x = a, a\in[x_{i}, x_{i+1}]$,此时这个不等式有上限:
		$$\max\limits_{x_{0}\le x \le x_{n}}|\omega_{n+1}(x)| \le |(a - x_{i})(a - x_{i+1})|h^{n - 1}p!q!$$
		其中$p + q = n + 1, p > 0, q > 0$,$p, q$为整数\\
		同时考虑两个限制,可以发现当$k = 0$或$k = n-1$且$x = \frac{x_{k+1} + x_{k}}{2}$有最大值上限估计:
		$$\max\limits_{x_{0}\le x \le x_{n}}|\omega_{n+1}(x)| \le \frac{1}{4}h^{n+1}n!$$\\\\
		\item[5.]设$L_{n}$为$n$次Legendre多项式,试证明:
		$$
		(1-x^{2})L_{n}'(x) +nxL_{n}(x) = nL_{n-1}(x).
		$$\\
		利用数学归纳法证明,假设当$\forall n, n \le k-1$时等式成立,则$n = k$时也成立,首先验证$k = 1, k = 2,k= 3$的情况:
		$$k = 1 \Rightarrow (1-x^2)+ x^2 = 1$$
		$$k = 2 \Rightarrow (1-x^2)3x + x(3x^2 - 1) = 2x$$
		$$k = 3 \Rightarrow (1-x^2)(15x^2-3) + 3x(5x^3 - 3x) = 9x^2 - 1$$
		显然$k=1,k=2, k = 3$均成立,则假设$n \le k - 1(k > 3)$时成立,则可以利用已知的递推式以及归纳假设:
		$$nL_{k}(x) = (2k - 1)xL_{k-1}(x) - (k - 1)L_{k-2}(x)$$
		两边对$x$求导,得到:
		$$kL_{k}'(x) = (2k-1)L_{k-1}(x) + (2k-1)xL_{k-1}'(x) - (k-1)L_{k-2}'(x)$$
		利用归纳假设替换$L_{k - 1}'(x), L_{k-2}'(x)$,得到:
		$$kL_{k}'(x)(1-x^2) = (2k-1)(1-kx^2)L_{k-1}(x)+3x(k-1)^2L_{k-2}(x) - (k-1)(k-2)L_{k-3}(x)$$
		利用递推公式替换$L_{k-2}(x)$,化简得到:
		$$kL_{k}'(x)(1-x^2) = (4k^2x^2-8kx^2+3x^2+2k - 1)L_{k-1}(x) - 3k(k-1)xL_{k}(x) - (k-1)(k-2)L_{k-3}(x)$$
		再两次利用递推公式替换$L_{k-3}(x)$,得到:
		$$kL_{k}'(x)(1-x^2) = (4k^2x^2-8kx^2+3x^2+2k - 1)L_{k-1}(x) - 3k(k-1)xL_{k}(x) - (2k-3)x((2k-1)xL_{k-1}(x) - kL_k(x))$$
		进一步化简得到:
		$$kL_{k}'(x)(1-x^2) =k^2L_{k-1}(x) - k^2xL_{k}(x) $$
		即:
		$$L_{k}'(x)(1-x^2) =kL_{k-1}(x) - kxL_{k}(x) $$
		证毕\\\\
		\item[6.]设$f\in C[0, 2\pi]$,且满足$f(0) = f(2\pi)$,记$S_{n} = span\{1,\cos x, \sin x, \cdots, \cos(nx), \sin(nx)\}.$\\
		(1)证明:生成$S_{n}$的函数组相互正交.\\
		(2)试求$f_{*} = arg \min\limits_{g\in S_{n}}||f-g||_{2}$,其中$||g||_{2} = (\int_{0}^{2\pi}g^2(x)dx)^{\frac{1}{2}}$\\\\
		(1)$\forall g_1(x), g_2(x) \in S_{n}$,分类讨论:\\
		当$g_1(x) = sin(mx), g_2(x) = sin(nx), m,n$为整数且$m, n > 0$时:
		$$\int_{0}^{2\pi}sin(mx)sin(nx)dx = \left\{ 
		\begin{array}{lcl}
		0 && m \ne n\\
		\pi && m = n
		\end{array}
		\right.$$
				当$g_1(x) = cos(mx), g_2(x) = cos(nx), m,n$为整数且$m, n > 0$时:
		$$\int_{0}^{2\pi}cos(mx)cos(nx)dx = \left\{ 
		\begin{array}{lcl}
		0 && m \ne n\\
		\pi && m = n
		\end{array}
		\right.$$
		当$g_1(x) = cos(mx), g_2(x) = sin(nx), m,n$为整数且$m, n > 0$时:
			$$\int_{0}^{2\pi}cos(mx)sin(nx)dx = 0$$
		当$g_1(x) = 1, g_2(x) = sin(nx), n$为整数且$n > 0$时:
		$$\int_{0}^{2\pi}sin(nx)dx = 0$$
		当$g_1(x) = 1, g_2(x) = cos(mx), m$为整数且$m > 0$时:
		$$\int_{0}^{2\pi}cos(mx)dx = 0$$
		当$g_1(x) = 1, g_2(x) = 1$时:
		$$\int_{0}^{2\pi}1dx = 2\pi$$
		所以可以证明生成$S_{n}$的函数组相互正交\\\\
		(2)我们可以猜想满足条件的$p(x) \in S_{n}$为原函数的傅里叶级数展开,然后在证明其满足最佳平方逼近的条件即可。
		构造函数
		$$p(x) = \frac{a_{0}}{2} + \sum_{n = 1}^{n}(a_{k}\cos (kx) + b_{k} \sin (kx))$$
		其中:
		$$a_{k} = \frac{1}{\pi}\int_{0}^{2\pi}f(x)cos(kx)dx , n= 0, 1, 2 \cdots$$
		$$b_{k} = \frac{1}{\pi}\int_{0}^{2\pi}f(x)sin(kx)dx , n= 1, 2 \cdots$$
		值得注意的是:
		$$a_{0} = \frac{1}{\pi}\int_{0}^{2\pi}f(x)dx = 0 $$
		则:
		$$p(x) = \sum_{n = 1}^{n}(a_{k}\cos (kx) + b_{k} \sin (kx))$$
		假设$\phi_{n}(x)$为任意$S_{n}$中的函数,且有$\phi_{0} = 1$,$r_n$为之前我们所构造得到的系数$a_n$与$b_n$的集合,由书中已经得出的结论如果我们所构造的函数满足:
		$$\sum_{j = 0}^{n}(\phi_{k}(x), \phi_{j}(x))r_{j} = (f(x), \phi_{k}(x)), k = 0, 1, \cdots, n$$
		就能证明$p(x)$为原函数的最小二次逼近,所以有:
		$$\sum_{j = 0}^{n}(\phi_{k}(x), \phi_{j}(x))r_{j} = \sum_{j=0}^{n} \int_{0}^{2\pi}\phi_{j}\phi_{k}\,dx \, (\frac{1}{\pi}\int_{0}^{2\pi}f(x)\phi_{k}(x)\,dx) $$
		观察我们在上一问中得到的结论:
		$$\int_{0}^{2\pi}\phi_{j}\phi_{k}\,dx = \left\{
		\begin{array}{lcl}
		\pi && j = k\quad j, k\ne 0,\\
		2\pi && j = k = 0\\
		0 && j \ne k
		\end{array}
		\right.$$
		则有
		$$\sum_{j = 0}^{n}(\phi_{k}(x), \phi_{j}(x))r_{j} = \int_{0}^{2\pi}f(x)\phi_{k}(x)\,dx$$
		满足最小二次逼近的要求,所以:
		$$f_{*}(x) = \sum_{n = 1}^{n}(a_{k}\cos (kx) + b_{k} \sin (kx))$$
		其中
		$$a_{k} = \frac{1}{\pi}\int_{0}^{2\pi}f(x)cos(kx)dx \quad n= 1, 2 \cdots$$
		$$b_{k} = \frac{1}{\pi}\int_{0}^{2\pi}f(x)sin(kx)dx \quad n= 1, 2 \cdots$$\\\\
		\item[7.]已知实验数据如下:
		\begin{table}[!htbp]
			\centering
			\begin{tabular}{c|ccccc}
				
				\hline
				$x_{i}$ & 19 & 25 & 31 & 38 & 44\\
				\hline
				$y_{i}$ & 19.0 & 32.3 & 49.0 & 73.3 & 97.8\\
				\hline
			\end{tabular}
		\end{table}
	使用MATLAB编程,用最小二乘法求形如$y = a + bx^{2}$的经验公式,并计算均方误差.\\\\
	假设我们所求的函数为:
	$$y(x) = a +bx^2$$
	则要求:
	$$R(a, b) = \sum_{i = 1}^{5}[y_{i} - y(x_{i})]^2 = \sum_{i = 1}^{5}(y_{i} - a - bx_{i}^2)^2$$
	取得最小值,即:
	$$\frac{\partial R(a,b)}{\partial a} = 0$$
	$$\frac{\partial R(a,b)}{\partial b} = 0$$
	由此得到线性方程组:
	$$\left\{
	\begin{array}{lcl}
	a + (\sum\limits_{i = 1}^{5}x_{i}^2)b = \sum\limits_{i = 1}^{5}y_{i}\\
	(\sum\limits_{i = 1}^{5}x_{i}^2)a+(\sum\limits_{i = 1}^{5}x_{i}^4)b = \sum\limits_{i = 1}^{5}x_{i}^2y_{i}
	\end{array}
	\right.$$
	可以利用MATLAB编程解此线性方程组得到解以及均方误差:
	\begin{lstlisting}[language=MATLAB]
		function res = lesq(x, y)
		a0 = 1;
		tmp = x.^2;
		a1 = sum(tmp);
		tmp1 = tmp.*y;
		tmp = tmp.^2;
		a3 = sum(tmp);
		a4 = sum(tmp1);
		a5 = sum(y);
		lhs = [a0, a1;a1, a3];
		rhs = [a5, a4]';
		res = lhs\rhs;
		error = sum((res(1) + res(2).*(x.^2) - y).^2);
		res(3) = error;
		end
	\end{lstlisting}
	最后的结果及均方误差为:
	$$\left\{
	\begin{array}{lcl}
	a = -0.3693\\
	b = 0.0510\\
	\delta = 1.9972
	\end{array}
	\right.$$
	\end{itemize}
\end{document}